% Remigiusz
\section{Wielomian Jonesa}
W tej sekcji zbadamy inny wielomianowy niezmiennik węzłów, wielomian Jonesa.
Został odkryty w 1984 roku przez Vaughana Jonesa, znajduje zastosowanie między innymi przy badaniu węzłów przemiennych.

\subsection{Nawias Kauffmana}
Zaczniemy od zdefiniowania nawiasu Kauffmana.
Przypomnijmy, wielomian Laurenta zmiennej $X$ to formalny symbol $f=a_r X^r + \ldots + a_s X^s$, gdzie $r, s, a_r, \ldots, a_s$ są całkowite i $r \le s$.

\begin{definicja}
	\emph{Nawias Kauffmana} $\langle D \rangle$ dla diagramu splotu $D$ to wielomian Laurenta zmiennej $A$, który jest niezmienniczy ze względu na gładkie deformacje diagramu, a przy tym spełnia trzy aksjomaty:
	\begin{enumerate}
		\item $\langle \NieWezel \rangle=1$
		\item $\langle D \sqcup \NieWezel \rangle = (-A^{-2} - A^2) \langle D \rangle$
		\item $\langle \PrawyKrzyz \rangle = A \langle \PrawyGladki \rangle + A^{-1} \langle \LewyGladki  \rangle$
	\end{enumerate}
\end{definicja}

Tutaj $\NieWezel$ oznacza standardowy diagram dla niewęzła, $D \sqcup \NieWezel$ jest diagramem, który powstaje z $D$ przez dodanie nieprzecinającej go krzywej zamkniętej, zaś trzy symbole $\PrawyKrzyz$, $\PrawyGladki$ oraz $\LewyGladki $ odnoszą się do diagramów, które są identyczne wszędzie poza małym obszarem.
Diagramy $\PrawyGladki$ oraz $\LewyGladki$ nazywa się odpowiednio dodatnim (prawym) i ujemnym (lewym) wygładzeniem $\PrawyKrzyz$

\begin{lemat}
	Nawias Kauffmana dowolnego diagramu można wyznaczyć w skończonym czasie.
\end{lemat}

\begin{proof}
	Jeżeli diagram $D$ ma $n$ skrzyżowań, to nieustanne stosowanie aksjomatu trzeciego pozwala na zapisanie $\langle D \rangle$ jako sumy $2^n$ składników, z których każdy jest po prostu zamkniętą krzywą i ma trywialny nawias ($\langle \MalyNieWezel \rangle = 1$).
	Nawias sumy wyznacza się korzystając z drugiego aksjomatu.
\end{proof}

Przedstawimy teraz wpływ ruchów Reidemeistera na nawias Kauffmana.

\begin{lemat}
	Pierwszy ruch Reidemeistera zmienia nawias Kauffmana zgodnie z poniższą regułą.
	Pozosałe ruchy Reidemeistera nie zmieniają nawiasu.
	\[
		% pierwszy ruch Reidemeistera
		\left\langle\begin{tikzpicture}[scale=0.05, baseline=-6]
			\clip (-12,-12) rectangle (1,7);
			\path[ARC] (-10,7) .. controls (-10,3) and (-10,0) .. (-6,-4);
			\path[ARC] (-6,0) .. controls (2,8) and (2,-10) .. (-6,-4);
			\path[ARC] (-10,-11) .. controls (-10,-8) and (-10,-5) .. (-9,-4);
		\end{tikzpicture}\right\rangle
		= -A^{-3}
		\left\langle\ \begin{tikzpicture}[scale=0.05,baseline=-6]
			\path[ARC] (-10,7) -- (-10,-11);
		\end{tikzpicture}\ \right\rangle
		\,\bullet\,
		% drugi ruch Reidemeistera
		\left\langle\begin{tikzpicture} [scale=0.04,baseline=-5]
			\path[ARC](0,10) .. controls (10,5) and (10,-9) .. (0,-14);
			\path[ARC] (10,10) .. controls (8,9) .. (7,8);
			\path[ARC] (3,4.5) .. controls (-1,0) and (-1,-4) .. (3,-8);
			\path[ARC] (10,-14) .. controls (8,-13) .. (7,-12);
		\end{tikzpicture}\right\rangle
		=
		\left\langle\ \begin{tikzpicture} [scale=0.04, baseline=-5]
		\path[ARC] (0,10) .. controls (3,8) and (3,0) .. (3,-2) .. controls (3,-4) and (3,-12) .. (0,-14);
		\path[ARC] (10,10) .. controls (7,8) and (7,0) .. (7,-2) .. controls (7,-4) and (7,-12) .. (10,-14);
		\end{tikzpicture}\ \right\rangle
		\,\bullet\,
		% trzeci ruch Reidemeistera
		\left\langle\begin{tikzpicture} [scale=0.04, auto, baseline=-6] %Reidemeister 3 left
			\path[ARC] (-10,10) -- (-6.6,6);
			\path[ARC] (-4,3) -- (10,-14);
			\path[ARC] (10,10) -- (6.6,6);
			\path[ARC] (4,3) -- (1.6,0);
			\path[ARC] (-1.6,-4) -- (-10,-14);
			\path[ARC] (-14,-2) .. controls (-6, -2) and (-6,8) .. (0,8);
			\path[ARC] (14,-2) .. controls (6, -2) and (6,8) .. (0,8);
		\end{tikzpicture}\right\rangle
		=
		\left\langle\begin{tikzpicture} [scale=0.04, auto, baseline=-6] %Reidemeister 3 right
			\begin{scope}[xshift=1300,rotate=180,yshift=110]
				\path[ARC] (-10,10) -- (-6.6,6);
				\path[ARC] (-4,3) -- (10,-14);
				\path[ARC] (10,10) -- (6.6,6);
				\path[ARC] (4,3) -- (1.6,0);
				\path[ARC] (-1.6,-4) -- (-10,-14);
				\path[ARC] (-14,-2) .. controls (-6, -2) and (-6,8) .. (0,8);
				\path[ARC] (14,-2) .. controls (6, -2) and (6,8) .. (0,8);
			\end{scope}
		\end{tikzpicture}\right\rangle.
	\]
\end{lemat}

\subsection{Spin}

\subsection{Wielomian Jonesa}

\subsection{Relacja kłębiasta}
Dotychczas wyznaczyliśmy wielomian Jonesa jedynie dla trywialnych splotów.
Spowodowane jest to tym, że chociaż nawias Kauffmana jest przydatny przy dowodzeniu różnych własności, to zupełnie nie nadaje się do obliczeń.
Dużo lepszym narzędziem jest następujące twierdzenie.

\begin{twierdzenie}[relacja kłębiasta]
	Wielomian Jonesa spełnia równość $V(\NieWezel) = 1$ oraz relację
	\[
		t^{-1} V(L_+) - tV(L_-) + (t^{-1/2} - t^{1/2}) V(L_0) = 0,
	\]

	gdzie $L_+$, $L_-$, $L_0$ to zorientowane sploty, kóre różnią się jedynie na małym obszarze: $\PrawyKrzyz$

\end{twierdzenie}
\subsection{Odwrotności, lustra i sumy}